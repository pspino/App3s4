\documentclass{article}
\usepackage[utf8]{inputenc}
\usepackage{hyperref}
\usepackage[english, french]{babel}

\begin{document}

\begin{titlepage}   
	\large{
		\begin{center}
			UNIVERSITÉ DE SHERBROOKE\\Faculté de génie\\
			Département de génie électrique et génie informatique\\
			\vspace{3cm}
			{\LARGE\textbf{Méchanique de réalité virtuelle}}\\
			\vspace{2cm}
			\LARGE{Rapport de l'App 3}\\
			\vspace{2cm}
			Présenté à\\l'équipe professorale de la session S4\\
			\vspace{2cm}
			Produit par\\Philippe Spino\\ Eric Beaudoin \\ Alexandre Gagnon
			\vspace{1cm}
			\vfill{07 Juin 2017 - Sherbrooke}
		\end{center}
	}
\end{titlepage}
\section{intro}
\noindent
Dans le cardre du cour de simulation de réalité virtuelle, le mandat de est simuler un vaisseau spacial s'écrasant sur un plantète extra-terrestre pour une scène d'un film d'animation. Le mandat se divisais en plusieurs tâches. Obtenir les variables d'états du système complex 
\section{Variables d'états}
\section{Calcul des équations des variables d'états }

\end{document}